
\section{Schwarzschild Spacetime}.

The Schwarzschild spacetime is given by the line element
\begin{equation}
	ds^2 = -\left( 1 - \frac{2M}{r}\right) dt^2 + \left( 1 - \frac{2M}{r}\right)^{-1} dr^2 + r^2 d\theta^2 + r^2 \sin^2 \theta d\phi^2.
\end{equation}

When compared with the standard form of the Kerr's line element in Boyer-Lindquist coordinates,
\begin{align}
	ds^2 = &-\left( 1- \frac{2Mr}{\Sigma} \right) dt^2 -\frac{4Mar\sin^2 \theta}{\Sigma} dt d\phi \nonumber  \\
	&+ \frac{\Sigma}{\Delta} dr^2 +\Sigma d\theta^2 + \sin^2 \theta \left( r^2 + a^2 +\frac{2Ma^2 r \sin^2 \theta}{\Sigma} \right) d\phi^2,
\end{align}
Schwarzschild's metric is obtained by taking $a=0$ and
\begin{align}
	\Sigma &= r^2 \\
	\Delta &= r^2 -2Mr.
\end{align}

Therefore, the potentials in the description of a particle moving in this spacetime reduce to
\begin{align}
	R &= E^2 r^4 - (r^2 -2Mr) \left[ r^2 + L^2 + Q \right] \\
	\Theta &= Q -  \frac{\cos^2 \theta}{\sin^2 \theta}L^2
\end{align}

Then, we have the function
\begin{align}
	\Xi &= E^2 r^4 - (r^2 -2Mr) \left[ r^2 + L^2 + Q \right] + (r^2 -2Mr) \left[ Q -  \frac{\cos^2 \theta}{\sin^2 \theta}L^2 \right] \nonumber \\
	\Xi &= E^2 r^4 - (r^2 -2Mr) ( r^2 + L^2 ) - (r^2 -2Mr) \frac{\cos^2 \theta}{\sin^2 \theta}L^2 \nonumber \\
	\Xi &= E^2 r^4 - r^4 - L^2 r^2 + 2Mr^3 + 2ML^2r - r^2 \frac{\cos^2 \theta}{\sin^2 \theta}L^2 
	  + 2Mr \frac{\cos^2 \theta}{\sin^2 \theta}L^2 
\end{align}

In order to write the equations of motion we need the derivatives
\begin{align}
\frac{\partial \Xi}{\partial E} &= 2E r^4 \\
\frac{\partial \Xi}{\partial L} &= - 2(r^2 -2Mr)L - 2(r^2 -2Mr) \frac{\cos^2 \theta}{\sin^2 \theta}L \nonumber \\
						&= - 2(r^2 -2Mr)L \left[1 +  \frac{\cos^2 \theta}{\sin^2 \theta} \right] \nonumber \\
						&= - 2(r^2 -2Mr)L \csc^2 \theta  \\
\end{align}
and also
\begin{align}
\frac{\partial}{\partial r}\left( \frac{\Delta}{2\Sigma}\right) &= \frac{\partial}{\partial r}\left( \frac{1}{2} - \frac{M}{r} \right) = \frac{M}{r^2} \\
\frac{\partial}{\partial r}\left( \frac{1}{2\Sigma}\right) &= \frac{\partial}{\partial r}\left( \frac{1}{2r^2}\right) = -\frac{1}{r^3}\\
\frac{\partial}{\partial r}\left( \frac{\Xi}{2\Delta \Sigma}\right) &= \frac{\partial}{\partial r}\left( \frac{E^2 r^4 - (r^2 -2Mr) ( r^2 + L^2 ) - (r^2 -2Mr) \frac{\cos^2 \theta}{\sin^2 \theta}L^2}{2r^2(r^2 -2Mr)}\right) \nonumber \\
&= \frac{1}{2} \frac{\partial}{\partial r}\left( E^2\left(1-\frac{2M}{r}\right)^{-1} - 1 - \frac{L^2}{r^2} - \frac{\cos^2 \theta}{\sin^2 \theta} \frac{L^2}{r^2} \right) \nonumber \\
&= -\frac{E^2}{\left(1-\frac{2M}{r}\right)^2}\frac{M}{r^2}  + \frac{L^2}{r^3} + \frac{\cos^2 \theta}{\sin^2 \theta} \frac{L^2}{r^3} \nonumber \\
&= -\frac{E^2 M}{(r-2M)^2}  + \frac{L^2}{r^3} \csc^2 \theta
\end{align}

\begin{align}
\frac{\partial}{\partial \theta}\left( \frac{\Delta}{2\Sigma}\right) &= \frac{\partial}{\partial r}\left( \frac{1}{2} - \frac{M}{r} \right) = 0 \\
\frac{\partial}{\partial \theta}\left( \frac{1}{2\Sigma}\right) &= \frac{\partial}{\partial r}\left( \frac{1}{2r^2}\right) = 0\\
\frac{\partial}{\partial \theta}\left( \frac{\Xi}{2\Delta \Sigma}\right) &= \frac{\partial}{\partial \theta}\left( \frac{E^2 r^4 - (r^2 -2Mr) ( r^2 + L^2 ) - (r^2 -2Mr) \frac{\cos^2 \theta}{\sin^2 \theta}L^2}{2r^2(r^2 -2Mr)}\right) \nonumber \\
&= \frac{1}{2} \frac{\partial}{\partial \theta}\left( -  \frac{\cos^2 \theta}{\sin^2 \theta} \frac{L^2}{r^2}\right) \nonumber \\
&= \cot \theta \csc^2 \theta \frac{L^2}{r^2} 
\end{align}

Using this expressions, the equations of motion of a particle in this spacetime are given by the Hamilton's equations

\begin{align*}
	\dot{t} &= \frac{1}{2r^2(r^2 -2Mr)} 2E r^4 = \frac{E r^2}{(r^2 -2Mr)}  \\
	\dot{r} &= p_r \left(1 - \frac{2M}{r} \right)  \\
	\dot{\theta} &= \frac{p_\theta}{r^2}\\
	\dot{\phi} &= - \frac{1}{2r^2(r^2 -2Mr)} \left[ - 2(r^2 -2Mr)L \csc^2 \theta \right] = \frac{L}{r^2 \sin^2 \theta}\\	
\end{align*}

\begin{align*}
\dot{p}_t &= 0\\
\dot{p}_r &= - \frac{M}{r^2}p_r^2 +  \frac{p_\theta^2}{r^3} -\frac{E^2 M}{(r-2M)^2}  + \frac{L^2}{r^3 \sin^2 \theta} \\
\dot{p}_\theta &= \frac{\cos \theta}{\sin^3 \theta} \frac{L^2}{r^2}\\
\dot{p}_\phi &= 0\\	
\end{align*}

