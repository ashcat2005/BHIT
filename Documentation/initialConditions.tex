
\chapter{Initial Conditions}.\\

In order to calculate the image of a black hole we will trace back the path of photons from a point in the observer's image plane, located at a distant point, onto the emission structure around the black hole. The image plane is considered as a grid and the received photons will have a momentum vector orthogonal to this plane.\\

\section{Coordinate Systems}
We will take Cartesian coordinates $(X,Y,Z)$ at the image plane of the observer and Cartesian coordinates $(x,y,z)$ centered at the black hole. These two systems are related  by a rotation and a translation. As seen in Figure XXX we may define a rotation between the system $(X,Y,Z)$ and an intermediate system $(X',Y',Z')$ as
\begin{equation}
\begin{pmatrix}
X' \\
Y' \\
Z'
\end{pmatrix} =
\begin{pmatrix}
\cos \alpha & 0 &-\sin\alpha \\
0 & 1 &0\\
\sin \alpha &  0 & \cos \alpha\\
\end{pmatrix} 
\begin{pmatrix}
X\\
Y\\
Z
\end{pmatrix}
\end{equation}

If the observer is located at a distance $D$ from the black hole as in Figure XXX, we introduce the spatial translation from the intermediate system $(X',Y',Z')$ to the system $(x,y,z)$,

\begin{equation}
\begin{pmatrix}
x \\
y \\
z
\end{pmatrix} =\begin{pmatrix}
X' \\
Y' \\
Z'
\end{pmatrix} +
\begin{pmatrix}
D \cos \alpha \\
0 \\
D \sin \alpha
\end{pmatrix}
\end{equation}

The composition of these two transformations give

\begin{equation}
\begin{pmatrix}
x \\
y \\
z
\end{pmatrix} =
\begin{pmatrix}
\cos \alpha & 0 &-\sin\alpha \\
0 & 1 &0\\
\sin \alpha &  0 & \cos \alpha\\
\end{pmatrix} 
\begin{pmatrix}
X\\
Y\\
Z
\end{pmatrix} +
\begin{pmatrix}
D \cos \alpha \\
0 \\
D \sin \alpha
\end{pmatrix}
\end{equation}
or more explicitly,

\begin{equation}
\begin{cases}
x &= X\cos \alpha - Z \sin\alpha + D \cos \alpha\\
y &= Y\\
z &= X \sin \alpha + Z \cos \alpha + D \sin \alpha\\
\end{cases}
\end{equation}

Since the angle $\alpha$ is related with the inclination angle by $\alpha = \frac{\pi}{2} - i$ we replace the trigonometric functions in the coordinate transformations by
\begin{equation}
\begin{cases}
x &= X\sin i - Z \cos i + D \sin i\\
y &= Y\\
z &= X \cos i + Z \sin i + D \cos i\\
\end{cases}
\end{equation}
or better
\begin{equation}
\begin{cases}
x &= (X+D)\sin i - Z \cos i \\
y &= Y\\
z &= (X+D) \cos i + Z \sin i \\
\end{cases}
\end{equation}

Since the black hole metric will be given in spherical coordinates, we introduce the relation between the Cartesian system $(x,y,z)$ and the coordinates $(r,\theta, \phi)$ by

\begin{equation}
\begin{cases}
r &= \sqrt{x^2 + y^2 + z^2} \\
\theta &= \arccos \left( \frac{z}{r}\right)\\
\phi &= \arctan \left( \frac{y}{x} \right) \\
\end{cases}
\end{equation}

Hence, the complete coordinate transformation needed to describe the initial conditions of the photon is 

\begin{equation}
\begin{cases}
r &= \sqrt{(X+D)^2 + Y^2 + Z^2} \\
\theta &= \arccos \left( \frac{(X+D) \cos i + Z \sin i}{\sqrt{(X+D)^2 + Y^2 + Z^2}}\right)\\
\phi &= \arctan \left( \frac{Y}{(X+D)\sin i - Z \cos i} \right) \\
\end{cases} \label{CoordinateTransformation}
\end{equation}

\subsection{Initial Position of a photon}

Consider a photon registered at the observer's image plane at the coordinates $(X,Y,Z) = (0, \alpha, \beta)$ at the time $t_0 = 0$. The spherical coordinates of this photon as seen from the black hole are given by the equation (\ref{CoordinateTransformation}) as

\begin{equation}
\begin{cases}
t_0 &= 0\\
r_0 &= \sqrt{D^2 + \alpha^2 + \beta^2} \\
\theta_0 &= \arccos \left( \frac{D \cos i + \beta \sin i}{\sqrt{D^2 + \alpha^2 + \beta^2}}\right)\\
\phi_0 &= \arctan \left( \frac{\alpha}{D\sin i - \beta \cos i} \right) \\
\end{cases} 
\end{equation}

\section{Momentum of a Photon}

Photons received by the distant observer will be considered to arrive perpendicular to the image plane. According to the orientation of the coordinates $(X,Y,Z)$ in Figure XXX, the Cartesian components of the incident photon are given by the 4-momentum vector
\begin{equation}
\tilde{k}^\nu_0 =\left( \tilde{k}^t_0, \tilde{k}^X_0, \tilde{k}^Y_0, \tilde{k}^Z_0 \right) = \left( K_0, K_0, 0, 0 \right).
\end{equation}

The components of the 4-momentum in the spherical coordinate system $(r,\theta,\phi)$ are obtained using the transformation law
\begin{equation}
k^\mu = \frac{\partial x^\mu}{\partial \tilde{x}^\nu} \tilde{k}^\nu.
\end{equation}

The relevant terms in the transformation matrix are obtained from equation (\ref{CoordinateTransformation}) as

\begin{equation}
\left. \frac{\partial r}{\partial X} \right|_{(0,\alpha,\beta)}= \left.\frac{X+D}{\sqrt{(X+D)^2 + Y^2 + Z^2}}\right|_{(0,\alpha,\beta)}
= \frac{D}{\sqrt{D^2 + \alpha^2 + \beta^2}} = \frac{D}{r_0}
\end{equation}

\footnotesize
\begin{equation}
\left. \frac{\partial \theta}{\partial X} \right|_{(0,\alpha,\beta)} = \left.-\frac{1}{\sqrt{r^2-[(X+D)\cos i + Z \sin i]^2}}  \left[ \cos i - \frac{(X+D)^2 \cos i + Z(X+D) \sin i}{r^2}\right]\right|_{(0,\alpha,\beta)} \nonumber
\end{equation}
\normalsize

\begin{align}
\left. \frac{\partial \theta}{\partial X} \right|_{(0,\alpha,\beta)} &= -\frac{1}{\sqrt{r^2_0 -[D\cos i + \beta \sin i]^2}}  \left[ \cos i - \frac{D^2 \cos i + D\beta \sin i}{r^2_0}\right]\\
\left. \frac{\partial \theta}{\partial X} \right|_{(0,\alpha,\beta)} &= -\frac{1}{\sqrt{\alpha^2 +[D\sin i - \beta \cos i]^2}}  \left[ \cos i - \frac{D^2 \cos i + D\beta \sin i}{r^2_0}\right]
\end{align}


\begin{align}
\left. \frac{\partial \phi}{\partial X} \right|_{(0,\alpha,\beta)} &= \left.-\frac{Y \sin i}{[(X+D)\sin i - Z \cos i ]^2 + Y^2 } \right|_{(0,\alpha,\beta)} \nonumber \\
\left. \frac{\partial \phi}{\partial X} \right|_{(0,\alpha,\beta)} &= -\frac{\alpha \sin i}{\alpha^2 +[D \sin i - \beta \cos i]^2 }  
\end{align}

Hence, the spatial components of the initial 4-momentum in spherical coordinates are

\begin{equation}
\begin{cases}
k^r_0 &= \frac{D}{r_0} K_0\\
k^\theta _0 &= -\frac{1}{\sqrt{\alpha^2 +[D\sin i - \beta \cos i]^2}}  \left[ \cos i - \frac{D^2 \cos i + D\beta \sin i}{r^2_0}\right] K_0 \\
k^\phi _0 &= -\frac{\alpha \sin i}{\alpha^2 +[D \sin i - \beta \cos i]^2 } K_0
\end{cases} \label{initialSpatialMomentum}
\end{equation}

The temporal component of the initial 4-momentum of the photon is obtained from the condition 
\begin{equation}
k^\mu k_\mu =  \eta_{\mu \nu} k^\mu k^\nu = 0,
\end{equation}
which gives
\begin{equation}
k^t_0 = \sqrt{(k^r_0)^2 + r_0^2 (k^\theta _0)^2 + r_0^2 \sin^2 \theta_0 (k^\phi _0)^2}. \label{initialTempMomentum}
\end{equation}

\section{Complete set of Initial Conditions}

The complete set of initial conditions are given by equations (\ref{CoordinateTransformation}), (\ref{initialSpatialMomentum}) and (\ref{initialTempMomentum}),

\begin{align}
t_0 &= 0\\
r_0 &= \sqrt{D^2 + \alpha^2 + \beta^2} \\
\theta_0 &= \arccos \left( \frac{D \cos i + \beta \sin i}{\sqrt{D^2 + \alpha^2 + \beta^2}}\right)\\
\phi_0 &= \arctan \left( \frac{\alpha}{D\sin i - \beta \cos i} \right) \\
k^t_0 &= \sqrt{(k^r_0)^2 + r_0^2 (k^\theta _0)^2 + r_0^2 \sin^2 \theta_0 (k^\phi _0)^2}\\
k^r_0 &= \frac{D}{r_0} K_0\\
k^\theta _0 &= -\frac{1}{\sqrt{\alpha^2 +[D\sin i - \beta \cos i]^2}}  \left[ \cos i - \frac{D^2 \cos i + D\beta \sin i}{r^2_0}\right] K_0 \\
k^\phi _0 &= -\frac{\alpha \sin i}{\alpha^2 +[D \sin i - \beta \cos i]^2 } K_0
\end{align}