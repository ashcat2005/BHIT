
\chapter{Geodesic Equations}.\\
Consider a general line element of  a general stationary and axially symmetric spacetime described by the line element
\begin{equation}
ds^2 = g_{tt} dt^2 + 2g_{t \phi} dt d\phi + g_{rr} dr^2 + g_{\theta \theta} d\theta^2 + g_{\phi \phi} d\phi^2.
\end{equation}

The Lagrangian describing the geodesic motion of a test-particle in this spacetime is
\begin{equation}
\mathcal{L} = \frac{1}{2} g_{\mu \nu} \dot{x}^\mu \dot{x}^\nu, 
\end{equation}
where $\dot{x}^\mu = \frac{dx^\mu}{d\lambda}$ and $\lambda$ is an affine parameter of the geodesic. 
The momentum
From the stationary and axially symmetric conditions of spacetime, we conclude that the metric components are independent of the coordinates $t$ and $\phi$. Therefore, there are two constants of motion: the specific energy at infinity, $E$, and the axial component of the specific angular momentum at infinity, $L_z$. These are defined as
\begin{align}
-E &= p_t = \frac{\partial \mathcal{L}}{\partial \dot{t}} = g_{tt} \dot{t} + g_{t \phi} \dot{\phi} \\
L_z &= p_\phi = \frac{\partial \mathcal{L}}{\partial \dot{\phi}} = g_{t\phi} \dot{t} + g_{\phi \phi} \dot{\phi}.
\end{align}












in Boyer-Lindquist coordinates,
\begin{align*}
	ds^2 = &-\left( 1- \frac{2Mr}{\Sigma} \right) dt^2 - -\frac{4Mar\sin^2 \theta}{\Sigma} dt d\phi\\
	&+ \frac{\Sigma}{\Delta} dr^2 +\Sigma d\theta^2 + \sin^2 \theta \left( r^2 + a^2 +\frac{2Ma^2 r \sin^2 \theta}{\Sigma} \right) d\phi^2,
\end{align*}
\begin{align*}
	\Sigma &= \Sigma(r,\theta)\\
	\Delta &= \Delta(r).
\end{align*}

The equations of motion of a particle in this spacetime are 

\begin{align*}
	\dot{t} &= \frac{1}{2\Delta \Sigma} \frac{\partial \Xi}{\partial E}\\
	\dot{r} &= \frac{\Delta}{\Sigma} p_r \\
	\dot{\theta} &= \frac{p_\theta}{\Sigma}\\
	\dot{\phi} &= - \frac{1}{2\Delta \Sigma} \frac{\partial \Xi}{\partial L}\\	
\end{align*}

\begin{align*}
\dot{p}_t &= 0\\
\dot{p}_r &= -\frac{\partial}{\partial r}\left( \frac{\Delta}{2\Sigma}\right) p_r^2 - \frac{\partial}{\partial r}\left( \frac{1}{2\Sigma}\right) p_\theta^2 + \frac{\partial}{\partial r}\left( \frac{\Xi}{2\Delta \Sigma}\right) \\
\dot{p}_\theta &= -\frac{\partial}{\partial \theta}\left( \frac{\Delta}{2\Sigma}\right) p_r^2 - \frac{\partial}{\partial \theta}\left( \frac{1}{2\Sigma}\right) p_\theta^2 + \frac{\partial}{\partial \theta}\left( \frac{\Xi}{2\Delta \Sigma}\right)\\
\dot{p}_\phi &= 0\\	
\end{align*}

\begin{align*}
	\Xi &= R + \Delta \Theta \\
	R &= P^2 - \Delta \left[ r^2 + (L-aE)^2 + Q \right] \\
	P &= E(r^2 + a^2) - aL\\
	\Theta &= Q - \cos^2 \theta \left[ a^2 (1-E^2) + \frac{L^2}{\sin^2 \theta} \right]
\end{align*}






